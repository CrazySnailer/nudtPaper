\chapter{绪论}
本章主要介绍本课题的研究背景、研究现状、主要研究内容以及论文结构。

\section{研究背景}
目前,内嵌了各种各样功能丰富的传感器的智能手机已经成为人们日常生活中集通信、感知及计算于一体的移动平台。通过内嵌的各种传感器如GPS、加速度、麦克风、光线以及磁力传感器等可以随时随地感知和获取人们自身及其周围环境的各种信息,通过对这些信息的处理可以了解人们的日常交互关系\upcite{eagle2006reality}、情绪\upcite{rachuri2010emotionsense}、行为模式\upcite{miluzzo2007cenceme}、活动状况\upcite{dong2011modeling}、 交通状况\upcite{thiagarajan2009vtrack}以及环境噪音\upcite {kanjo2010noisespy} 等。其中,通过智能手机所收集各种数据研究人们之间的日常交互行为和人们之间的社会关系成为普适计算领域的一个重点研究的问题,针对这一问题,研究人员开展了大量的研究工作,并取得了一系列的研究成果。RealityMining \upcite{eagle2006reality}基于手机所收集的各种数据推理人们之间的社会交互关系以及群组的活动韵律,从而洞察个人和组织的行为模式;CenceMe\upcite{miluzzo2007cenceme}利用智能手机获取的GPS位置以及加速度等感知数据推理用户的当前状态包括行为活动、情绪、习惯以及朋友之间的交互情况,并将当前的状态发布到社交网站Facebook 上与朋友进行共享;fMRi\upcite{aharony2011social}重点研究分析了家庭和朋友圈对个体行为在社交网络中所受的影响;StudentLife\upcite{wang2014studentlife}重点研究了在校学生的日常活动、交互情况与学生的精神健康以及学业成绩之间的关系;文献\cite{stopczynski2014measuring}则从多渠道、细粒度地收集各种反映在校学生日常活动和交互情况的各种数据,从多个层面真实、全面地反映学生日常活动以及他们之间的交互行为和交互关系。但是,这些研究重点关注的是人们之间的日常交互行为和交互关系,并没有涉及人们之间的关系强度。关系强度度量的是人们之间的亲密程度,通过关系强度,我们可以更好地了解人们之间的关系的强弱,了解人们之间的亲密程度,进而可以更好地预测社会关系的演变、社交结构的变化、促进信息传播以及传染疾病的预防与控制等。
\par 目前,智能手机已经成为人们日常生活当中不可或缺的通信交流工具,通过智能收集可以随身随地的获取位置、通话记录、短信、微信等体现人们之间日常交互和社会关系的各种信息,人们之间的交互频率、时间、位置、地点、距离以及轨迹相似性等信息能够直接体现人们之间的交互关系以及关系强度,因为关系密切的人们之间更愿意面对面地进行交流,而且朋友之间尤其是好友之间会经常碰面如聚合、去餐馆、一起游览等等,通过对这些信息的分析处理,可以更好地度量朋友之间的关系强度。

\section{研究现状}
\subsection{社会关系强度的研究现状}
社会关系强度理论从Mark Granovetter对弱关系的研究\upcite{granovetter1973strength}开始,Burt在弱关系理论的基础上提出了结构洞理论\upcite{burt2009structural}。Granovetter等\upcite{granovetter1973strength}认为弱关系和强关系的测量应该考虑四个维度,即两个用户之间的互动频率、用户感情的投入程度、两个用户之间的关系亲密程度以及两个用户在互惠互利上的交换程度;Wegner 等\upcite{carruthers2007illusion}指标化了这四个维度,从而可以使用数值来度量强关系和弱关系;Muncer等\upcite{burrows2000virtual}认为关系的数量和交往的频率对关系强度有重要的影响;Paolillo等\upcite{coles1999monoclonal}发现亲密程度和语言交流中非正式称呼出现的次数有直接关系。随着众多学者对关系强度的研究不断深入,逐渐形成了从互动频率、联系次数和亲密程度这三方面作为关系强度核心测量指标的主流研究观点\upcite{petroczi2007measuring}。

\subsection{基于移动数据的社会关系研究现状}
Eagle等\upcite{pentland2009inferring}招募了100名志愿者,记录他们手机上的蓝牙信息,以此为基础,通过分析志愿者之间的相遇次数来研究志愿者之间是同事关系或朋友关系。Miluzzo等\upcite{miluzzo2007cenceme}使用诺基亚智能手机N95开发了一款名为Cenceme的应用,该应用能够实时采集手机中传感器产生的数据,通过对采集的传感器数据的分析,研究用户所处的状态以及任意两个用户之间的关系。Mtibaa等\upcite{mtibaa2008you}收集了28个参加同一个计算机会议的参会者的蓝牙数据,并利用对这些数据的分析结果构建了社会网络关系图。Crandall等\upcite{crandall2008feedback}尝试利用不同用户同时在相同地理位置的出现次数来推断社交链接。他们研究发现即使两个用户在同一地理位置的同时出现次数比较少,这两个用户也有很大可能相互认识。

\subsection{基于移动轨迹数据关系强度的研究现状}
物理层次的交互,比如面对面的交流更有可能反映社会关系的真实状态。但是到目前为止,这方面研究还比较少,文献\cite{ma2014effective}首次使用移动轨迹数据来度量用户之间的关系强度。文献\cite{ma2014effective}提出了一个层级熵关系度量方法HERMA(Hierarchical Entropy-based Relationship Measurement Approach)。HERMA利用对从轨迹中抽取的位置共现记录的分析来推断用户之间的物理交互,进而使用用户之间的物理交互来度量用户之间的社会关系强度。HERMA设计了一个层级区域结构来对用户的位置共现记录建模。并在此基础上,HERMA进一步提出用户熵和区域熵分别来度量用户的活跃性和区域开放性。最后在仿真数据集上对该方法进行了实验验证。

\section{研究内容}
本课题针对如何通过轨迹数据度量用户之间的关系强度问题展开研究,力求设计一个能够同时处理基站数据和GPS数据的通用框架模型。希望能从多个角度利用轨迹数据来度量用户之间的关系强度,经过调研和研究,决定从轨迹空间距离和基于主题模型的行为模式相似度两方面来度量用户之间的关系强度。课题中,为了达到所期望的目标,从以下三个方面展开研究。
\par (1)如何对GPS数据进行处理以及标记语义标签
\par GPS原始数据存在较大误差,且用户日常活动存在大量路上的点,而实际只需要考虑用户停留在宿舍、实验室等语义位置相关的点,故需要对GPS数据进行降噪并剔除路上的点。在此基础上,需要通过一些方法发现与GPS原始数据对应的如宿舍、实验室等语义位置,进而对每个语义位置标记对应的语义标签。在本课题中,通过对各种滤波算法进行试验,发现分段卡尔曼滤波有非常好的降噪效果;通过对GPS数据的进一步分析,发现路上点的密度远小于用户处于语义位置时的点的密度,因此采用基于密度的异常点剔除方法,且该方法可以自动学习参数;当前该领域用来发现语义位置的聚类算法存在一些问题,比如需要预先知道类别的个数或者对参数比较敏感,本科题决定采用最新提出的一个基于密度的聚类算法\upcite{rodriguez2014clustering}来发现语义位置,该方法对参数更鲁棒,且不需要预先知道类别个数;在得到语义位置的基础上,需要通过一些方法匹配语义位置对应的语义标签,目前通用的方法是人工手动标注,经过分析发现可以通过反地理编码,语义标签推断以及输入自动补全来减少用户和语义标签标注系统的交互。

\par (2)如何使用轨迹数据来度量用户之间的关系强度
\par a)基于轨迹空间距离的关系强度度量方法
\par 用户轨迹的相似程度在一定程度上反映了两个用户之间的亲密程度,本文尝试使用基于轨迹空间距离的方法来度量用户之间的关系强度。以往的研究中较多学者使用欧氏距离,编辑距离,DTW等方法来研究轨迹之间的相似度。但是,存在的一个问题是,欧氏距离只能用来度量两个长度相等的轨迹序列的相似度,通过实验发现使用编辑距离计算得到的相似度效果不是很理想,而DTW更偏向于序列长度较长的序列。在此基础上,对DTW的相似度度量结果使用三种方法归一化。并且发现用户每天活动的多样性不同使得该天轨迹数据的相似度对最终的相似度贡献不同,因此,使用用户每天轨迹序列的熵值对用户每天的相似度加权,实验证明该方法确实能够得到一个更好的结果。
\par b)基于主题模型的行为模式相似度度量方法
\par 通过分析研究发现,用户语义位置、语义标签均可以视为自然语言处理中的单词,而用户每天的语义位置序列和语义标签序列均可以视为自然语言处理中的句子,进而每个用户全部的语义位置序列和语义标签序列均可视为自然语言处理中的文档。因此可以从语义位置和语义标签这两个层次使用自然语言处理中一些度量文档相似度的方法来度量用户在不同层次行为模式的相似度。本课题使用LDA主题模型和word2vec模型分别来度量用户行为模式的相似度。并分析了不同的主题个数和不同的向量长度对实验结果的影响。
\par c)如何对三层实验结果进行融合
\par 前文分别研究了如何通过轨迹空间距离的方法度量用户之间的关系强度,如何通过基于语义位置的用户行为模式相似度度量用户之间的关系强度,以及如何通过基于语义标签的用户行为模式相似度度量用户之间的关系强度。在三层实验结果的基础上,如何对三层实验结果进行融合是一个很重要的问题。轨迹空间距离计算结果为两个用户轨迹的距离,距离越小说明两个用户关系强度越强,而基于行为模式的方法计算用户行为模式的相似度,相似度越大说明两个用户关系强度越强,因此不能通过简单的加权求和作为最终的关系强度,且因为数据量太少,无法通过训练来确定加权的权值。参考集成学习的思想,首先对三层方法的计算结果按关系强度进行排序,然后对和该用户关系第k 强的朋友进行投票,若某个朋友出现两次及以上则认为该朋友与该用户关系第k强,否则取基于轨迹空间距离的方法度量的结果作为最终结果。通过实验证明,该投票方法进一步改善了该模型的效果。
\par 本课题针对如何度量日常生活中人们之间的关系强度问题展开研究,提出了一个既可以对GPS数据进行处理又可以对基站数据进行处理,从日常轨迹、语义位置以及语义标签三个层次度量人们之间关系强度的层级模型URSHV(User Relationship Strength Hierarchy Vote) 。该模型首先采用DTW模型通过计算用户之间的空间距离来度量用户轨迹之间的相似度,进而使用轨迹序列熵值对用户每天轨迹的相似度进行加权处理,并将其作为用户之间的关系强度;其次,采用主题模型LDA分别计算用户之间的基于语义位置和语义标签的行为模式的相似性,将其作为用户之间的关系强度;最后,采用集成学习的思想对三个层次的度量结果进行投票,以投票结果作为最终的用户之间的关系强度。用户和陌生人之间的关系强度因为互相不认识应该为0,即用户和朋友之间的关系强度应该大于用户和陌生人之间的关系强度。但是对用户来说,虽然一些陌生人不认识,但是经常在一些地方同时出现,因此使用轨迹不能完全反映用户和所有人的关系强度,只考虑用户和其好友之间的关系强度。
\par (3)如何对实验结果进行评估以及不同参数对模型的影响
\par 本课题实验结果为用户的全部朋友按计算得到的关系强度由强到弱组成的朋友序列,实际结果是用户的全部朋友按实际关系强度由强到弱组成的朋友序列,如何度量两个包含相同元素的有序序列的一致性是本课题很重要的一个问题。本课题设计的模型中包含大量的参数,而这些参数究竟对模型有什么样的影响也是本课题很重要的一个问题。所以,在本课题中,首先基于有序序列逆序对的概念提出一种称为一致性评分的度量方法来度量两个包含相同元素的有序序列的一致性;在此基础上,通过实验观察并分析本课题设计的模型中包含的每一个参数对一致性评分的影响来研究不同参数对模型的影响。

\section{论文结构}
本文的论文结构由六章组成,各章内容概述如下:
\par 第一章为绪论,主要描述了课题的研究背景引出本文主要研究动因,然后从社会关系强度、基于移动数据的社会关系、基于移动轨迹数据关系强度的研究现状进行了分别描述,在此基础上,提出了课题的三个主要研究内容,最后给出了论文的组织结构。
\par 第二章的主要内容为相关技术研究,描述了三方面与课题工作密切相关的技术。首先从滤波算法和聚类算法两方面描述了轨迹数据预处理的主要技术,并对其进行分析;其次分析了时间序列相似度能够在一定程度上反映用户之间的关系强度,并描述了时间序列相似度度量方法中的两种主流算法,编辑距离和DTW,以及序列熵值的计算方法;最后分析了自然语言处理方法对轨迹数据处理的有效性以及主题模型和用户行为模式之间的关系,并描述了两种常用的自然语言处理模型,LDA主题模型和word2vec神经语言模型。
\par 第三章描述了用户关系强度层级投票模型的整体框架图,主要包括SASLL系统,该系统包括对GPS数据的处理以及语义标签标注技术;用户关系强度计算方法。这两个模块将在第四章和第五章分别进行描述。
\par 第四章主要描述了如何处理GPS数据以及语义标签标注技术。首先描述了SASLL(System Annotating Semantic Label of Location)系统框架;其次从降低数据噪声、剔除异常点、聚类得到语义位置三个方面描述了如何计算得到对应的语义位置;最后描述了如何发现新位置以及如何计算新位置的语义标签提示。
\par 第五章描述了用户关系强度的计算方法。首先描述了模型框架;其次分别描述了轨迹数据的处理与准备、语义位置数据的处理与准备以及语义标签数据的处理与准备三种方法;最后分别描述了基于轨迹数据的关系强度计算、基于主题模型的关系强度计算以及结果投票三个方法。
\par 第六章首先描述了实验用到的数据集;其次描述了对实验结果的评估方法;最后展示了实验结果并进行分析。
\par 第七章对本课题作出总结,并对接下来的工作作出展望。
