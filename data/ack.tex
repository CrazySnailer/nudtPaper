
%%% Local Variables:
%%% mode: latex
%%% TeX-master: "../main"
%%% End:

\begin{ack}
\par 时光荏苒,白驹过隙,求学三载,感触良多。恐思别离,废书而叹,秉笔直书,以谢众恩,忆似水年华,感时光不复。时至今日,余具所得,跪而叩谢者复四也。
\par 一谢师恩,感谢我的硕士导师史殿习教授和刘东波研究员,感谢史老师对我生活上的关心和科研上的指导。史老师从我还在本科四年级的时候就时常和我联系,为我解答学习上的困惑。作为对国防科学技术大学知之甚少同时又胸怀无限憧憬的学生,我心中怀揣着许多疑问和顾虑。史老师针对我的担忧一一为我进行了详细的解答,当了解到国防科学技术大学对于地方生的管理非常严格,甚至连出校门都需要请假的时候,我表现出了极度的恐惧,史老师这时候一方面给我详细讲述了学校的规章制度,另一方面给我传授“严于律己,严谨学习”的理念让我能够提前适应军校的生活学习环境。等我初到学校参加军训期间,因为刚刚开始接触学校的军事化管理心里有些抵触情绪外加身体原因导致个人情绪低落、消极。史老师在一次通话中察觉到我的情绪变化后立即找我谈心,为我解开心结。我本身是一个比较愚钝的人,学习新事物的速度都要比师门其他同学慢,承蒙史老师不放弃,给了我一些相关论文让我由浅入深的开展科研活动。同时也非常感谢史老师对我们的严格要求,因为史老师您的以身作则,要求您在实验室的时候我们就要在实验室,我们才将更多的课余时间利用在科研学习上而不是虚度年华。从我们进入课题前,史老师便要求我们每周做一次课题研究报告汇报研究进度,不断鞭策我们按时按质完成课题研究。同时史老师作为一名科研人员,在繁重的科研任务之余还涉猎广泛,每次进办公室看见史老师书柜中汗牛充栋的书籍以及交谈中的引经据典,无时无刻不影响着我,不要一味的只读专业书籍,还需要广泛阅读其他书籍来扩充提升自己。在以后步入社会工作的日子里,我会更加努力,踏踏实实做人,勤勤奋奋做事,不辜负史老师的期望。同时我还要真诚的感谢尹刚老师、刘惠老师在我开题阶段对我的课题提出的宝贵的建议,使我更加清晰的认识到课题研究中的不住和需要努力的地方,使得我的课题研究能够顺利进行。

\par 二谢同门,非常感谢杨若松学长、李寒学长、吴渊学长、谭杰夫学长、樊泽栋学长、陈茜学姐在师门中对我的照顾,感谢您们让我迅速融入到学校和师门的环境中,杨若松学长每次都能够不厌其烦的为我讲解在算法上遇见的问题,李寒学长带着我跟着他的镜头踏遍了祖国大好河山,吴渊学长在我最开始学习传感器编程的时候给予我很多帮助,陈茜学姐陪我们聊人生聊理想,让我们相信生活不止当下还有诗和远方。感谢同窗好友陈晓鹏、赵邦辉、刘帆、莫晓赟、李中秋,感谢陈晓鹏教会我很多为人处世的道理,让我得到进一步的成长;感谢赵邦辉经常和我们把酒言欢释放心中的郁闷;感谢刘帆经常帮我们报账,为我们提供“后勤保障”;感谢晓赟和我分享研究中的问题,带领我们共同做移动感知数据收集框架;感谢中秋陪我谈心,将各地旅游的照片与我分享,带我走出实验室投向大自然的怀抱。感谢童哲航、王冉、颜丙政、魏菁、成瑶瑶学弟学妹,和你们在一起的日子总是那么轻松愉快,你们的青春活力给我这颗心脏注入了新的血液,我相信你们都是更加优秀的人,衷心的祝愿你们未来的道路越走越好,前途无量。感谢同舍室友李玉乐、高翔、衡冬冬,和你们生活在一起永远是那么欢乐,和你们讨论学术问题总是让我获益匪浅。

\par 三谢队干,感谢朱涛政委、孙友佳队长。感谢您们在我求学的这段时间对我在生活上的帮助,感谢您们每次假前教育的谆谆教导,感谢您们对我的帮助,我进入社会工作后,一定谨遵教导,做一位优秀的科大人!

\par 四谢父母,授我身体发肤,养育之恩无以为报,唯日后谨身节用,以养严君;他日同风起,扶摇九万里,以慰萱堂。
\par
\begin{center}
    \qquad 中入潭州已六霜,国学堂里铸鱼肠。\\
  砥志研思遨书海,指点江山瞰橘洲。\\
  莫叹虚负凌云志,且待水击三千里。\\
  即日叩别恩师去,一片丹心化碧江。
\end{center}

\end{ack}
