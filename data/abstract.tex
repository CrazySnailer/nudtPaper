\begin{cabstract}
近几年,随着传统手机逐渐被智能手机所取代,搭载了智能操作系统(如iOS、Android和Windows Phone) 的智能手机已经成为了人们生活中集日常通信、娱乐游戏、商务办公、感知计算等于一体的移动掌上终端平台。通过搭载了更多传感器设备的智能手机能够随时随地的获取到用户的位置、通信记录、短信记录、日常轨迹分布情况等各种体现用户与用户之间的隐藏社会关系的感知信息。人们之间的轨迹相似性、日常轨迹的共现频率和时长,用户连接Wi-Fi 的共现使用情况以及用户蓝牙之间的交互信息都能够分析得出人与人之间的交互关系以及他们之间的关系强度。通过对用户之间的相似性和关系强度的探究和计算有利于进一步发展个人的社交网络和探究社会群体结构的发展以及演变过程等。

传统的度量用户之间的社交关系大多采用的是基于社交关系数据来分析用户之间的社会关系(社交关系数据如通话记录,短信记录,社交软件的交互等),本文基于智能手机所采集的非社交关系数据针对如何使用非社交关系数据来分析度量人们社交活动之间的相似性以及关系强度问题展开了进一步的深入研究。设计并且实现了一个基于用户轨迹数据、Wi-Fi感知数据、蓝牙感知数据抽象出的多个层次的度量人与人之间在社交活动中相似性以及关系强度的计算模型RSMHD(Relationship Strength based on Multiple Hierarchy Dimension)。整个计算模型的主要涵括的内容如下:
\par 为了便于研究的顺利开展,我们假设陌生人之间的关系强度应该是无限趋近于零的,同时我们为了便于对计算结果的准确性进行判断和数据的采集,本文研究对象主要限于在校学生群体。社会心理的理论支持以及研究成果指出:相似的人更容易产生交集成为朋友,现实生活中人们也更加倾向于同自己相似的人做朋友,也就是说相似导致喜欢,从而发展为友情或者爱情。这也就预示着相似度人与人之间的关系强度要高于非相似着,这也就为我们的研究提供了理论依据。
\par 首先,针对用户日常轨迹数据我们分别从空间移动轨迹和现实语义轨迹出发,对用户之间的相似性进行计算,本文采用位置漂移修正算法对部分因无信号原因导致的GPS位置缺失的轨迹进行预测补全;针对补全后的用户轨迹,采用时间片式的卡尔曼滤波算法剔除用户GPS轨迹中的异常点;采用基于时间-密度的聚类算法得到用户空间轨迹中的停留点,在此基础上,采用基于语义规则的语义标签标注机制,通过用户参与,反地理编码等手段对停留点进行现实语义标注,将用户的空间轨迹转变为符合现实意义的日常语义轨迹,为利用位置感知数据来计算用户的关系强度做好准备。然后针对用户空间语义轨迹,本文采用了快速DTW计算方法来计算用户空间轨迹之间的相似性,并采用动态加权算法对计算结进行加权处理;针对日常语义轨迹,我们分别从两个层次出发,一方面采用了Word2vec来计算用户在切片时间内的语义轨迹的相似性;另一方面,根据用户的语义轨迹挖掘出用户的轨迹运动模式,通过快速计算相似度算法Simhash来计算出用户的日常轨迹运动模式之间的相似性并对结果采取加权处理得出用户之间的关系强度,最后对计算结果进行融合来得出用户的关系强度。

\par 其次,针对用户WiFi感知数据,采用关联拓扑图表示某个时刻的用户WiFi上下文环境信息,然后提取出用户WiFi的交互情况作为计算用户关系强度的特征之一;针对时间片上的拓扑图创新的提出了通过计算图与节点之间的相似性作为某时刻WiFi上下文环境信息之间的相似性。针对用户的蓝牙感知数据,通过结合脸呀交互时长、交互频率、蓝牙上下文环境相似性等来计算出用户之间的关系强度。

\par 最后,利用集成学习的思想对以上三种不同感知数据源计算结果进行融合处理,得到基于不同非社交数据源所计算得到的最终用户关系强度。在该模型算法研究的基础上,本文基于自主开发的用户上下文信息收集系统StarLog 收集了多名学生长时间的智能手机感知数据,利用该数据源来计算出用户之间的关系强度,并结合用户的相似度问卷调查表对结果进行有效性验证,表明该模型能够有效地利用非社交关系度量出用户之间的关系强度.
\end{cabstract}
\ckeywords{关系强度;非社交数据;轨迹模式;word2vec;集成学习}

\begin{eabstract}
Smart phones have become an integral part of daily life communication tools, we can collect intelligently location, call logs, text messages, WeChat anywhere which reflect a variety of information daily interactions and social relations between people. People interaction frequency, time, location, distance and the similarity of trajectory information reflects the strength of the relationship and the relationship between people.Relationship strength reflects the degree of intimacy between two different persons, which is of great importance in analyzing human's social relationship as well as social network.
\par In this paper, we propose a URSHV(User Relationship Strength Hierarchy Vote), which can measure the relationship between people in daily life through GPS data from three levels, namely: daily trajectory, semantic locations and semantic labels.To sum up, the main research contents and contributions are as follows:
\par First of all, the strength of the relationship between users and semantic location with semantic labels are closely related, this paper uses segmented kalman filtering algorithm on GPS trajectory data to de-noising; using the clustering algorithm based on density of position trajectory data clustering, and form a semantic position; on this basis, the semantic annotation mechanism based on Rules, the semantic annotation of semantic encoding, geographic location by anti semantic label inference and input auto completion etc; the GPS position trajectory data sequence clustering into meaningful semantic position and semantic labels, which laid the foundation for the semantic location and semantic labels based on the strength calculation of the relationship between users.
\par Secondly, in order to calculate the strength of the relationship between users from the three levels of trajectory data, semantic location and semantic labels, using DTW model of spatial distance calculation between users to measure the similarity between the users, the use of trajectory sequence similarity of users every day to track the entropy weighting processing, and the strength of relationship between users the LDA were calculated using the topic model; semantic locations similarity and semantic labels based on behavior patterns among users, as the strength of the relationship between users; measurement results of three levels of the ensemble learning theory to vote, to vote as the strength of relationship between end users.
\par Finally, on the basis of the above study, based on the MIT reality mining project of publicly available data sets, similarity between users by using the data set of the questionnaire, users construct between a real relationship strength as a benchmark for testing proposed an inverse logarithmic induced score measurement method to measure the strength of the relationship between users based on, and effectiveness model of URSHV for experiments, the results show that the model can effectively measure the strength of the relationship between users.

\end{eabstract}
\ekeywords{relationship strength; trajectory similarity; DTW; entropy; LDA; vote}

