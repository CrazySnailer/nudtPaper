\begin{cabstract}
最近几年,随着传统手机逐渐被智能手机所取代,搭载了智能操作系统(如iOS、Android和Windows Phone) 的智能手机已经成为了人们生活中集日常通信、娱乐游戏、商务办公、感知计算等于一体的移动掌上终端平台。通过搭载了更多传感器设备的智能手机能够随时随地的获取到用户的位置、通信记录、短信记录、日常轨迹分布情况等各种体现用户与用户之间的隐藏社会关系的感知信息。人们之间的轨迹相似性、日常轨迹的共现频率和时长,用户连接Wi-Fi的共现使用情况以及用户蓝牙之间的交互信息都能够分析得出人与人之间的交互关系以及他们之间的关系强度。通过对用户之间的相似性和关系强度的探究和计算有利于进一步发展个人的社交网络和探究社会群体结构的发展以及演变过程等。

基于以上社会心理的理论支持以及研究成果,我们考虑进一步通过从用户的轨迹上下文来度量人与人之间的关系强度。我们认为陌生人之间的关系强度应该是无限趋近于零的。但是在现实生活中,即使我们与某些陌生人并不相识,但是同样有可能出现在同一个位置场所,因此为了便于研究,本文的研究对象仅限于本校学生群体即用户及其相熟的人。本文从GPS 位置轨迹数据出发,针对如何利用智能手机收集的用户位置上下文信息分析度量人与人之间的关系强度展开研究。提出了一个基于用户日常轨迹数从地理位置、语义轨迹以及轨迹模式三个层次,每个层次又从多个维度出发的度量人与人之间的关系强度的度量模型RSMHD(Relationship Strength based on Multiple Hierarchy Dimension)。该模型与其它研究相比,采用了从计算用户轨迹运动模式出发,通过分析从用户日常轨迹中挖掘出的用户轨迹运动模式之间的相似性,进而计算用户之间的相似性,分析用户之间的关系强度的新的研究出发点;同时在计算基于用户语义轨迹的关系强度层次中采用了一种快速计算语义轨迹相似性的处理方法,有效地降低了计算的时间复杂度,提高了效率。最后基于课题收集的26名学生连续20 天所收集的手机传感器数据实现了RSMHD 模型。

传统的度量用户之间的社交关系大多采用的是基于社交关系数据来分析用户之间的社会关系(社交关系数据如通话记录,短信记录,社交软件的交互等),本文基于智能手机所采集的非社交关系数据针对如何使用非社交关系数据来分析度量人们社交活动之间的相似性以及关系强度问题展开了进一步的深入研究。设计并且实现了一个基于用户轨迹数据、Wi-Fi感知数据、蓝牙感知数据出发抽象出多个层次的度量人与人之间在社交活动中相似性以及关系强度。

本文针对如何度量日常生活中人们之间的关系强度问题展开研究,提出了一个既可以对GPS 数据进行处理又可以对基站数据进行处理,从日常轨迹、语义位置以及语义标签三个层次度量人们之间关系强度的层级模型URSHV(User Relationship Strength Hierarchy Vote)。概括起来,主要研究内容和贡献如下:
\par 首先,由于语义位置及语义标签与用户之间的关系强度密切相关,为此本文采用分段卡尔曼滤波算法对GPS位置轨迹数据进行降噪处理;采用基于密度的聚类算法对位置轨迹数据进行聚类,形成语义位置;在此基础上,采用基于规则的语义位置标注机制,通过反地理编码、语义标签推断以及输入自动补全等方式对语义位置进行语义标注;从而将GPS位置轨迹数据序列聚类成有意义的语义位置和语义标签,为后续的基于语义位置和语义标签计算用户之间的关系强度奠定了基础。
\par 其次,为了从位置轨迹数据、语义位置以及语义标签三个层次计算用户之间的关系强度,采用DTW模型计算用户之间的空间距离来度量用户日常轨迹之间的相似度,进而使用轨迹序列熵值对用户每天轨迹的相似度进行加权处理,并将其作为用户之间的关系强度;采用主题模型LDA 分别计算用户之间的基于物理位置和语义位置的行为模式的相似性,将其作为用户之间的关系强度;采用集成学习的思想对三个层次的度量结果进行投票,以投票结果作为最终的用户之间的关系强度。
\par 最后,在上述研究基础上,基于MIT的Reality Mining项目的公开的数据集,利用该数据集当中的用户之间调查问卷的相似度,构造用户之间真实的关系强度作为测试基准,提出一种基于逆序对数的一致性评分度量方法对用户之间的关系强度进行度量,进而对URSHV 模型的有效性进行实验验证,结果表明该模型能够有效地度量用户之间的关系强度。
\end{cabstract}
\ckeywords{关系强度;轨迹相似度;DTW;熵;LDA;投票}

\begin{eabstract}
Smart phones have become an integral part of daily life communication tools, we can collect intelligently location, call logs, text messages, WeChat anywhere which reflect a variety of information daily interactions and social relations between people. People interaction frequency, time, location, distance and the similarity of trajectory information reflects the strength of the relationship and the relationship between people.Relationship strength reflects the degree of intimacy between two different persons, which is of great importance in analyzing human's social relationship as well as social network.
\par In this paper, we propose a URSHV(User Relationship Strength Hierarchy Vote), which can measure the relationship between people in daily life through GPS data from three levels, namely: daily trajectory, semantic locations and semantic labels.To sum up, the main research contents and contributions are as follows:
\par First of all, the strength of the relationship between users and semantic location with semantic labels are closely related, this paper uses segmented kalman filtering algorithm on GPS trajectory data to de-noising; using the clustering algorithm based on density of position trajectory data clustering, and form a semantic position; on this basis, the semantic annotation mechanism based on Rules, the semantic annotation of semantic encoding, geographic location by anti semantic label inference and input auto completion etc; the GPS position trajectory data sequence clustering into meaningful semantic position and semantic labels, which laid the foundation for the semantic location and semantic labels based on the strength calculation of the relationship between users.
\par Secondly, in order to calculate the strength of the relationship between users from the three levels of trajectory data, semantic location and semantic labels, using DTW model of spatial distance calculation between users to measure the similarity between the users, the use of trajectory sequence similarity of users every day to track the entropy weighting processing, and the strength of relationship between users the LDA were calculated using the topic model; semantic locations similarity and semantic labels based on behavior patterns among users, as the strength of the relationship between users; measurement results of three levels of the ensemble learning theory to vote, to vote as the strength of relationship between end users.
\par Finally, on the basis of the above study, based on the MIT reality mining project of publicly available data sets, similarity between users by using the data set of the questionnaire, users construct between a real relationship strength as a benchmark for testing proposed an inverse logarithmic induced score measurement method to measure the strength of the relationship between users based on, and effectiveness model of URSHV for experiments, the results show that the model can effectively measure the strength of the relationship between users.

\end{eabstract}
\ekeywords{relationship strength; trajectory similarity; DTW; entropy; LDA; vote}

