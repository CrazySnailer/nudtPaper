\begin{cabstract}
近几年,随着传统手机逐渐被智能手机所取代,搭载了智能操作系统(如iOS、Android和Windows Phone) 的智能手机已经成为了人们生活中集日常通信、娱乐游戏、商务办公、感知计算等于一体的移动掌上终端平台。通过搭载了更多传感器设备的智能手机能够随时随地的获取到用户的位置、通信记录、短信记录、日常轨迹分布情况等各种体现用户与用户之间的隐藏社会关系的感知信息。人们之间的轨迹相似性、日常轨迹的共现频率和时长,用户连接Wi-Fi 的共现使用情况以及用户蓝牙之间的交互信息都能够分析得出人与人之间的交互关系以及他们之间的关系强度。通过对用户之间的相似性和关系强度的探究和计算有利于进一步发展个人的社交网络和探究社会群体结构的发展以及演变过程等。
\par 传统的度量用户之间的社交关系大多采用的是基于社交关系数据来分析用户之间的社会关系(社交关系数据如通话记录,短信记录,社交软件的交互等),社会心理的理论支持以及研究成果指出:相似的人更容易产生交集成为朋友,现实生活中人们也更加倾向于同自己相似的人做朋友,也就是说相似导致喜欢,从而发展为友情或者爱情。这也就预示着相似度人与人之间的关系强度要高于非相似者,本文基于智能手机所采集的在校学生群体的非社交关系数据针对如何使用非社交关系数据来分析度量人们社交活动之间的相似性以及关系强度问题展开了进一步的深入研究。设计并且实现了一个基于用户轨迹数据、Wi-Fi感知数据、蓝牙感知数据抽象出的多个层次的度量人与人之间在社交活动中相似性以及关系强度的计算模型RSMHD(Relationship Strength based on Multiple Hierarchy Dimension)。整个计算模型的主要涵括的内容如下:
%\par 为了便于研究的顺利开展,我们假设陌生人之间的关系强度应该是无限趋近于零的,同时我们为了便于对计算结果的准确性进行判断和数据的采集,本文研究对象主要限于在校学生群体。社会心理的理论支持以及研究成果指出:相似的人更容易产生交集成为朋友,现实生活中人们也更加倾向于同自己相似的人做朋友,也就是说相似导致喜欢,从而发展为友情或者爱情。这也就预示着相似度人与人之间的关系强度要高于非相似着,这也就为我们的研究提供了理论依据。
\par 首先,针对用户日常轨迹数据我们分别从空间移动轨迹和现实语义轨迹出发,对用户之间的相似性进行计算。针对用户轨迹,采用时间片式的卡尔曼滤波算法剔除用户GPS轨迹中的异常点;采用基于时间-密度的聚类算法得到用户空间轨迹中的停留点,在此基础上,采用基于语义规则的语义标签标注机制,通过用户参与,反地理编码等手段对停留点进行现实语义标注,将用户的空间轨迹转变为符合现实意义的日常语义轨迹,为利用位置感知数据来计算用户的关系强度做好准备。然后针对用户空间语义轨迹,本文采用了快速DTW计算方法来计算用户空间轨迹之间的相似性,并采用动态加权算法对计算结进行加权处理;针对日常语义轨迹,我们分别从两个层次出发,一方面采用了Word2vec来计算用户在切片时间内的语义轨迹的相似性;另一方面,根据用户的语义轨迹挖掘出用户的轨迹运动模式,通过快速计算相似度算法Simhash来计算出用户的日常轨迹运动模式之间的相似性并对结果采取加权处理得出用户之间的关系强度,最后对计算结果进行融合来得出用户的关系强度。

\par 其次,针对用户WiFi感知数据,采用关联拓扑图表示某个时刻的用户WiFi上下文环境信息,针对时间片上的拓扑图创新的提出了通过计算图与节点之间的相似性作为某时刻WiFi上下文环境信息之间的相似性。针对用户的蓝牙感知数据,我们借助和以上类似的处理通过蓝牙上下文环境相似性等来计算出用户之间的关系强度。

\par 最后,利用集成学习的思想对以上三种不同感知数据源计算结果进行融合处理,得到基于不同非社交数据源所计算得到的最终用户关系强度。在该模型算法研究的基础上,本文基于自主开发的用户上下文信息收集系统StarLog 收集了多名学生长时间的智能手机感知数据,利用该数据源来计算出用户之间的关系强度,并结合用户的相似度问卷调查表对结果进行有效性验证,表明该模型能够有效地利用非社交关系度量出用户之间的关系强度。
\end{cabstract}
\ckeywords{关系强度;非社交数据;轨迹模式;快速语义变换;集成学习}
\begin{eabstract}
Recent years, with the replacement of traditional mobile phones by smart phones, which were equipped with an intelligent operating system (such as IOS, Android and Windows Phone) has become a mobile computing platform of communication, Entertainment, business office and computing. Various sensors are embedded in the smartphones, so we can collect the data that can reflect the hidden social relations between the users such as user's location, communication records, SMS records, daily trajectories. It is great important to develop the social networks and explore the development and evolution of social group structures by find out the relationship strength among people.
\par Most of researches on social relationships and relationship strength between two users are based on social-relations data (social-relations data means call records, SMS records, the interaction of social network software, etc.). In this research, we propose a REMHD (Relationship Strength based on Multiple Hierarchy Dimension), which focus on non-social data to analysis the relationship strength between people thought daily trajectories,WiFi data and Bluetooth data. Our works in this research and contributions are mainly reflected as follows:

\par Firstly, we use timeslice kalman filtering algorithm to remove the noising points in GPS data; finding out the stay points and using the clustering algorithm to detect the clustering of GPS data or stay points by density and time, and marking the semantic labels with points; on this basis, we use DTW algorithm to calculate the similarity between user space trajectories. As for semantic trajectories of users, we put forward a two dimensions method, one the one hand, we use quickly simhash algorithm to calculate user`s strength with semantic trajectories; one the other hand, after mining out the user's trajectory movement pattern, we can get the strength thought user's trajectory movement pattern. Thus we merge above dimensions.
\par Secondly, the association topology is used to represent the WiFi sensors data at every piece of time. A new method is proposed to calculate the relationship strength based on WiFi data, we regard the similarity between two WiFi topology graphs as the similarity between different WiFi context information at the same time, which is the same as strength calculation via Bluetooth data.
\par Finally, on the base of above results, ensemble learning is used to merge the measurement results of three levels as the final relationship strength between users. We develop a context information collection system StarLog to collected a long term sensors data of students to calculate the relationship between students, and effectiveness RSMHD for experiments, the results indicate that the RSMHD can effectively detect the relationship strength between users.
\end{eabstract}
\ekeywords{relationship strength; non-social data; trajectory movement pattern;  quickly simhash algorithm; ensemble learning}

